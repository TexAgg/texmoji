% Autogenerated on \VAR{today()}.

\documentclass{article}
\usepackage[utf8]{inputenc}
\usepackage{texmoji}
\usepackage{verbatim}
\usepackage{multicol}
\usepackage{listings}
\usepackage{hyperref}

\title{The \texttt{texmoji} Package}
\author{Matt Gaikema \\ \href{mailto:mgaikema1@gmail.com}{mgaikema1@gmail.com} \\ \href{http://www.mattgaikema.com/}{www.mattgaikema.com}}
\date{\today}

\begin{document}

\maketitle

\tableofcontents

\section{Introduction}
Ever thought your PhD thesis was lacking pizzaz?
Or your paper needed more excitement?
Enter TeXMoji, emojis for LaTeX.

TeXMoji is generated using by scraping the \href{http://unicode.org/emoji/charts/full-emoji-list.html}{full emoji database} and saving the images, 
and then generating the \texttt{sty} and \texttt{tex} files using templates.
The source code used for the scraping and generating can be found at \href{https://github.com/TexAgg/texmoji}{github.com/TexAgg/texmoji}.

\section{Reference}
To use a specific emoji, find its number on the \href{http://unicode.org/emoji/charts/full-emoji-list.html}{full emoji database}
and use the command \verb|\texmoji{<n>}|, where \verb|n| is its number.

\begin{multicols}{2}
\begin{itemize}
\BLOCK{for key, value in data.iteritems()}
	\item \verb|\texmoji{\VAR{key}}|: \texmoji{\VAR{key}}
\BLOCK{endfor}
\end{itemize}
\end{multicols}

\section{License}
\texttt{texmoji} is licensed under the terms of the GNU General Public License.

\lstinputlisting[breaklines=true]{LICENSE}

\end{document}